\documentclass[11pt]{article}
\usepackage{fancyhdr}
\usepackage{color}
\usepackage{multicol}
\usepackage{fancyvrb}
\usepackage{enumitem}
\usepackage{graphicx}
\usepackage{sectsty}
\usepackage{amsmath}
\usepackage{amssymb}
\usepackage{hyperref}
\usepackage{array}
\newcommand{\sectionbreak}{\clearpage}

\usepackage{tikz}
\usepackage{tkz-euclide}
\usetkzobj{all}

\allsectionsfont{\centering}

\usepackage{draftwatermark}
	\SetWatermarkText{\copyright wolf-math.com}
	\SetWatermarkScale{4}
	\SetWatermarkLightness{1}

\usepackage[margin=1in, headsep=0pt]{geometry}
\setlength{\parindent}{0cm}
\pagestyle{empty}

\begin{document}
wolf-math.com\\

\section*{Systems of Linear Functions}

\subsection*{Goals}

\textbf{SWBAT} solve systems of equations by graphing.\\

\textbf{SWBAT} solve systems of equations by substitution.\\

\textbf{SWBAT} solve systems of equations by elimination.\\

\subsection*{Standards} 

\textbf{Creating Equations \hfill A-CED}\\

Create equations that describe numbers or relationships\\

3.	 Represent constraints by equations or inequalities, and by systems of
equations and/or inequalities, and interpret solutions as viable or non-
viable options in a modeling context. For example, represent inequalities
describing nutritional and cost constraints on combinations of different
food\\

\subsection*{Connections}

\textbf{Before} we learned about graphing lines and linear functions.  \\

\textbf{Now} we are learning about what happens when many linear function come together.\\

\textbf{Later} we will learn what happens when two linear functions are multiplied; we get a quadratic function.\\

\let\stdsection\section
\renewcommand\section{\newpage\stdsection}

\section*{System of Equations}

\subsection*{Solving by Graphing}

\textbf{Systems of Equations:} A set of equations that you deal with all at once.
One method of finding the solution to a system of equations is to find where their graphs overlap.\\

Example: 

\begin{equation*}
  \begin{cases}
	y=-2x+5\\
	
	y=\frac{2}{3}x-\frac{1}{3}\\
  \end{cases}
\end{equation*}
	
If we graph both of these equations we get the following graph.\\

\begin{center}
%\includegraphics[scale=.4]{systems1.jpg}\\
\end{center}

We can see that the lines overlap at the point $(2,1)$. This is the solution to our system.

\hrulefill

\textbf{You Try:} Solve the system of equations by graphing.\\

\begin{multicols}{2}

\begin{equation*}
	\begin{cases}
		y=-x+2\\
		y=x-2	
	\end{cases}
\end{equation*}

	
\begin{center}
%\includegraphics[scale=.4]{graph1.jpg}
\end{center}

\begin{equation*}
	\begin{cases}
		y=\frac{1}{2}x-2\\
		y=-3x+5	
	\end{cases}
\end{equation*}

	
\begin{center}
%\includegraphics[scale=.4]{graph1.jpg}
\end{center}

\end{multicols}
	
\pagebreak

\textbf{No Solution:} If two lines have the same slope that means that they will never intersect, which means there is no solution. 

\begin{multicols}{2}
\begin{equation*}
	\begin{cases}
			y=\frac{5}{3}x-3\\

			y=\frac{5}{3}x+2
	\end{cases}
\end{equation*}

%\includegraphics[scale=.5]{graph1.jpg}\\


\begin{equation*}
	\begin{cases}
			y=\frac{2}{5}x+1\\

			y=\frac{2}{5}x+2
	\end{cases}
\end{equation*}

%\includegraphics[scale=.5]{graph1.jpg}\\



\end{multicols}

\pagebreak

\textbf{Infinite Solutions:} If two lines overlap eachother, then there is infinite solutions. They're the same line! Don't get this confused with \textbf{No Solution} because both of them have the same slope.\\

\begin{equation*}
	\begin{cases}
			y=\frac{1}{2}x+1\\

			 y=\frac{1}{2}x+1\\
	\end{cases}
\end{equation*}

\begin{center}
%\includegraphics[scale=.5]{graph1.jpg}
\end{center}

\pagebreak

\begin{center}
	\begin{Large}
		Solving Systems by Substitution\\
	\end{Large}
\end{center}

Systems of equations can be solved using a common variable, and setting them equal to one another. 

\begin{equation*}
	\begin{cases}
		y=-x+2\\
		y=x-2	
	\end{cases}
\end{equation*}


Since both equations begin with $y$, they can be set equal to one another. 

$$-x+2=x-2$$

Now just solve for $x$

$$2x=4$$\ $$x=2$$\

Plug the solution $x=2$ into one of the equations to solve the system\\

$$y=(2)-2=0$$

so the final answer is $(2,0)$

\hrulefill

\textbf{You Try:} Solve the system of equations\\

\begin{equation*}
	\begin{cases}
		y=\frac{1}{2}x-2\\
		y=-3x+5	
	\end{cases}
\end{equation*}

\vspace{1cm}

$$\frac{1}{2}x-2=-3x+5$$


\pagebreak

\textbf{Solve the Systems of Equations} by graphing.\\

\begin{enumerate}
	
	\item \begin{equation*}
		\begin{cases}
			y=3x-2\\
			y=-3x+2	
		\end{cases}
	\end{equation*}

\vspace{2in}

	\item \begin{equation*}
		\begin{cases}
			y=4x-1\\
			y=-x+4	
		\end{cases}
	\end{equation*}
	
\vspace{2in}

	\item \begin{equation*}
		\begin{cases}
			y=-\frac{3}{2}x-4\\
			y=\frac{1}{2}x+4
		\end{cases}
	\end{equation*}
	


\end{enumerate}

\section*{Solving by Elimination}

Given a system of equations, we can solve them by elimination. This means looking at the equations \textit{vertically} rather than \textit{horizontally}. Usually when we solve systems by elimination it is when the equations are in \textbf{Standard Form}, that means that they are not in \textbf{Slope Intercept Form}.\\

\textbf{Example:}\\

$\begin{cases}
	2x+y=9\\
	3x-y=16
\end{cases}$

\vspace{.2in}

Since there is a single $y$ for both of these, I can eliminate them by adding them together (the first one is positive and the second one is negative, so adding will cancel them out)\\


$
	2x+y=9\\
	+\\	
	3x-y=16\\
\underline{\hspace{1in}}\\
	5x+0y=25
$ 

Since there is a $0y$ that means that it has been \textit{ELIMINATED}. So we can solve for $x$.\\

$5x=25 \rightarrow x=5$\\

Plug in $5$ for $x$ to obtain $y$. $y=-1$

\hrulefill

\textbf{Example 2:} Sometimes we need to do some manipulation.\\

$\begin{cases}
	4x-3y=25\\
	-3x+8y=10
\end{cases}$

\vspace{.2in}

Nothing cancels, but we can multiply by a \textit{scaler} to make one of the numbers the same. Not unlike finding a common denominator.\\

$\begin{cases}
	(3)(4x-3y=25)\\
	(4)(-3x+8y=10)
\end{cases}$

\vspace{.2in}

$\begin{cases}
	12x-9y=75\\
	-12x+32y=40
\end{cases}$

\vspace{.2in}

Now the $12x$ cancels with the $-12x$. So I can add them as...\\

$23y=115 \rightarrow y=5$\\

Now plug $5$ into the equation for $y$ to find the value of $x$\\

$x=10$\\

\pagebreak

\section*{Solving Systems Review}

NAME\underline{\hspace{3in}}



\textbf{Solve the system by graphing}\\

\begin{multicols}{2}

\begin{equation*}
	\begin{cases}
		y=-\frac{4}{3}x+2\\
		y=\frac{1}{3}x-3	
	\end{cases}
\end{equation*}

	
\begin{center}
%\includegraphics[scale=.4]{graph1.jpg}
\end{center}

\begin{equation*}
	\begin{cases}
		y=\frac{1}{2}x-2\\
		y=\frac{1}{2}x+1	
	\end{cases}
\end{equation*}

	
\begin{center}
%\includegraphics[scale=.4]{graph1.jpg}
\end{center}

\begin{equation*}
	\begin{cases}
		y=2x-4\\
		y=\frac{1}{3}x+1	
	\end{cases}
\end{equation*}

	
\begin{center}
%\includegraphics[scale=.4]{graph1.jpg}
\end{center}

\begin{equation*}
	\begin{cases}
		y=\frac{9}{5}x-5\\
		
		y=\frac{10}{11}x-5	
	\end{cases}
\end{equation*}

	
\begin{center}
%\includegraphics[scale=.4]{graph1.jpg}
\end{center}

\end{multicols}

\pagebreak

\hrulefill

\textbf{Solve the system using substitution}\\

\begin{multicols}{2}

$\begin{cases}
	y=x\\
	y=-x-2
\end{cases}$

\vspace{.3in}

$\begin{cases}
	y=\frac{1}{4}x-1\\
	y=2x+19
\end{cases}$

\vspace{.3in}

$\begin{cases}
	y=-x+5\\
	y=-\frac{1}{4}x+1
\end{cases}$

\vspace{.3in}

$\begin{cases}
	y=\frac{2}{3}x+4\\
	y=-\frac{1}{3}x+3
\end{cases}$

\vspace{.3in}

\end{multicols}

\hrulefill

\textbf{Solve the system using ELIMINATION}\\

\begin{multicols}{2}

$\begin{cases}
	x-2y=14\\
	x+3y=9
\end{cases}$

\vspace{.3in}

$\begin{cases}
	4x+3y=-1\\
	5x+4y=1
\end{cases}$

\vspace{.3in}

$\begin{cases}
	4x-y=10\\
	2x+3y=12
\end{cases}$

\vspace{.3in}

$\begin{cases}
	x+3y=-5\\
	4x-y=6
\end{cases}$

\vspace{.3in}

\end{multicols}

\hrulefill

\textbf{Word Problem:} The admission fee at a small fair is \$1.50 for children and \$4.00 for adults. On a certain day, 2200 people enter the fair and \$5050 is collected. How many children and how many adults attended?

\pagebreak %THIS IS THE NEXT LESSON!!!

\section*{Solving Systems Quiz}

\vspace{12pt}

NAME\underline{\hspace{3in}}\\


\textbf{Solve the system by graphing}\\

\begin{multicols}{2}

\begin{equation*}
	\begin{cases}
		y=-\frac{1}{2}x+2\\
		y=\frac{1}{4}x-1	
	\end{cases}
\end{equation*}

	
\begin{center}
%\includegraphics[scale=.4]{graph1.jpg}
\end{center}

\begin{equation*}
	\begin{cases}
		y=-\frac{2}{3}x+4\\
		y=-\frac{1}{3}x+3	
	\end{cases}
\end{equation*}

	
\begin{center}
%\includegraphics[scale=.4]{graph1.jpg}
\end{center}

\begin{equation*}
	\begin{cases}
		y=2x-3\\
		y=x+1	
	\end{cases}
\end{equation*}

	
\begin{center}
%\includegraphics[scale=.4]{graph1.jpg}
\end{center}

\begin{equation*}
	\begin{cases}
		y=3x-6\\
		
		y=x	
	\end{cases}
\end{equation*}

	
\begin{center}
%\includegraphics[scale=.4]{graph1.jpg}
\end{center}

\end{multicols}

\pagebreak

\hrulefill

\textbf{Solve the system using substitution}\\ % SECOND PAGE Group quiz

\begin{multicols}{2}

$\begin{cases}
	y=\frac{3}{4}x-9\\
	y=-3x+6
\end{cases}$

\vspace{1in}

$\begin{cases}
	y=\frac{3}{4}x-10\\
	y=2x+19
\end{cases}$

\vspace{1in}

$\begin{cases}
	y=-3x+7\\
	y=\frac{3}{2}x+4
\end{cases}$

\vspace{1in}

$\begin{cases}
	y=x-1\\
	y=-x+1
\end{cases}$

\vspace{1in}

\end{multicols}

\hrulefill

\textbf{Solve the system using ELIMINATION}\\

\begin{multicols}{2}

$\begin{cases}
	2x+3y=20\\
	-2x+y=4
\end{cases}$

\vspace{1in}

$\begin{cases}
	7x+4y=2\\
	9x-4y=30
\end{cases}$

\vspace{1in}

$\begin{cases}
	x-2y=14\\
	x+3y=9
\end{cases}$

\vspace{1in}

$\begin{cases}
	3x-2y=2\\
	5x-5y=10
\end{cases}$

\vspace{1in}

\end{multicols}

\hrulefill

\textbf{Extra Credit:} A boat traveled 210 miles downstream and back. The trip downstream took 10 hours. The trip back
took 70 hours. What is the speed of the boat in still water? What is the speed of the current?

\end{document}